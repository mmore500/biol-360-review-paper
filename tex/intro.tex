\section{Introduction} \label{sec:introduction}

As popularly conceived, evolutionary biology stands somewhat apart from many other branches of science.
Significant effort among evolutionary biologists focus on developing an understanding for how contemporary diversity of life came to be.
The historical contingency of the subject matter makes some questions addressed by evolutionary biology very difficult to explore through reproducible experimentation to attempt to falsify currently held hypotheses, the primary tool of scientific inquiry \cite{Smith2016ExplanationsBiology}.
For example, it would be difficult to test, in the traditional experimental fashion, the hypothesis that hairlessness evolved in hominid precursors due to natural selection for its capability to facilitate evaporative cooling.
That is, it would be near impossible to conduct alternate trials under different conditions to observe if the hypothesis is sufficient to explain the outcome.
This is not to say that empirical evidence does not exist for these claims: it does, and in overwhelming proportions.
In fact, in several cases evolutionary theory predicted yet-undiscovered empirical evidence, such as the amphibian-fish intermediate fossil \textit{Tiktaalik roseae} \cite{Daeschler2006APlan}.

Nonetheless, repeatable experiments are a desirable piece of the scientific puzzle of evolutionary biology.
Post hoc explanations for a phenomenon suffer inherent bias.
Fit between hypothesis and observation in traditional experimental work is interpreted as strong evidence for the validity of the explanation.
However, fit of the hypothesis and observation cannot be interpreted as strong evidence for the validity of the hypothesis due to the ``powerful psychological mechanism of hindsight bias'' \cite{Smith2016ExplanationsBiology}.
Essentially, ex post facto explanations always match the evidence they describe because they were created exactly to do so.
From this well is drawn the accusation of the ``just-so story'' commonly leveled against aspects of evolutionary biology \cite{Smith2016ExplanationsBiology}

Repeatable experiments have been employed to great effect in other scientific endeavors to understand natural history, like geology and cosmology.
This work has been present throughout the history of evolutionary biology; for example, the development of modern genetics has provided a firm underpinning for the notion of heritable variation necessary to evolutionary theory.
It was through repeatable experiments, such as Mendel's work with peas or the Hershey-Chase experiments, through which this aspect of our understanding of evolution was established \cite{Griffiths2015IntroductionAnalysis}.
Experimental work continues to serve as a foundation of modern evolutionary biology.
The work of the Bradshaw group exemplifies experimental work performed in the field.
This group uses experimental techniques to develop a detailed and concrete account of aspects of the evolutionary story, establishing, for example, how specific genetic changes in the California Monkeyflower manifest in phenotypic changes that, in turn, observably affect pollinator behavior in such a way as to promote reproductive isolation \cite{Byers2014FloralMimulus}.
Such experimental work in evolutionary biology tends to focus in on a specific component of the evolutionary process such as the genetic basis of heritable phenotypic variation or the manner in which phenotypic traits interact with the environment in regards to selection.
At most, experimental evolution tracks the evolutionary changes in response to experimental treatments over tens or hundreds of generations.
Famous examples of such efforts include long-term evolutionary experiments with the common fruit fly \textit{Drosophila melanogaster} \cite{Rose1984ArtificialMelanogaster}, unicellular yeast \textit{Saccharomyces cerevisiae} \cite{Ratcliff2012ExperimentalMulticellularity.}, and experiments observing \textit{E. coli} over hundreds of generations \cite{ATWOOD1951PeriodicColi.}.
Performing experiments to delve into evolution on a larger scale is difficult, expensive, and, critically, time-consuming.
Such experiments are inherently rate-limited by the generation-time of their experimental subjects.

It is into this relatively underexploited and important scientific terrain --- evolutionary biology experiments that unfold over vast stretches of generation on generation unfolding --- that the Lenski laboratory launched its Long Term Evolution Experiment (LTEE) in the year 1988 \cite{Lenski2017TheSite}. 
The aim of this work was to better understand how natural selection, historical constraint, and serendipity contribute to the adaptation of species and divergence between species observed in biology.
It was specifically hoped to observe long-term trajectories of mean fitness and genetic variation in fitness within and between groups sharing a common ancestor evolving in parallel \cite{Lenski1991Long-TermGenerations}.
In order to assess these questions in a timely and economical fashion, the experiment was carried out with a bacterial strain, \textit{E. coli}.
As of March of 2017, the Lenski experiment had surpassed 67,000 successive generations of experimental \textit{E. coli} evolution.
The project has generated rich theoretical insight into the questions it was conceived to address.
Moreover, it has generated a rich biological library of \textit{E. coli} that were suspended at regular intervals throughout the evolutionary process.
The biological materials generated by the LTEE have been leveraged by other researchers to investigate questions ranging from DNA supercoiling \cite{Crozat2010ParallelColi} to population divergence resulting in the occupation of a new ecological niche \cite{Plucain2014EpistasisColi}.
The LTEE has attracted a fair amount attention, both from scholarly and popular audiences \cite{Dawkins2009TheEvolution, Zimmer2009MicrocosmLife, Zimmer2011TortoiseFlask, Lenski2016EvolutionaryLenski}.
This notoriety stems, in part, from the feats of endurance and tireless exacting execution that were required to successfully sustain the LTEE over several decades.
The project is also recognized for the scientific value that the experiment has yielded, and continues to yield, with the elapse of generation upon generation of laboratory-reared \textit{E. coli}.
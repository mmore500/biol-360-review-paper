\section{Conclusion} \label{sec:conclusion}

The Lenski Long Term Evolution Experiment exquisitely demonstrates the unique opportunities for insight afforded by experimental evolution.
Studying the process of evolution in the laboratory allows for unparalleled repeatability, control over experimental conditions, and an ability for holistic measurement and observation.
As discussed in the preceding sections, these factors have enabled detailed study of specific evolutionary events that took place over the course of the LTEE, which has shed light on big questions, such as the evolutionary roles of historical contingency and mutational modulators.
As detailed in Section \ref{sec:introduction}, despite its notoriety the LTEE is by no means unique in its pursuit of experimental evolution.
In 2010, the Lenski laboratory became affiliated with a larger coallition to to study ``evolution in action,'' the NSF-funded BEACON Center \cite{Lenski2017TheSite}.
In addition to \textit{in vivo} work such as the Lenski LTEE, the BEACON Center was founded to support experimental evolution work both \textit{in silico}.
Excitingly, hybrid \textit{in silico}-\textit{in vivo} research, such as work to understand the relationship between population structure and evolutionary innovation, trialled in digital simulation then confirmed in the petri dish \cite{Nahum2015ABacteria.}, has also begun to emerge from the Center. 
Given the unique perspective provided by experimental evolution, the continuing LTEE and other experimental evolution projects that have begun to flourish at the BEACON center promise to yield exciting evolutionary insight far into the future.

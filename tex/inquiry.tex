\section{Results} \label{sec:results}

As was discussed in Section \ref{sec:introduction}, the Long Term Evolution Experiment has allowed researchers to address a broad set of questions of interest to evolutionary biology.
We will briefly discuss just three here.
First, Section \ref{sec:potentiation} discusses the question the experiment originally set out to investigate: the role of historical contingency in evolution.
The evolution in one of the replicate lineages of a capacity to metabolize citrate aerobically, an unexpected phenomenon, has provided great insight into this domain. 
Two other unexpected phenomenon, the evolution of traits that significantly altered mutation rates and the emergence and long-time coexistence of distinct sub-lineages in one of the LTEE lineages, have shed light on questions of evolvability and niche specialization. 
These topics are discussed in Sections \ref{sec:evolvability} and \ref{sec:niche} respectively.

\subsection{Potentiation} \label{sec:potentiation}

To the surprise, and initial suspicion, of LTEE researchers, in 2003 --- when approximately 33,000 generations of experimental evolution had elapsed --- the lineage Ara-3 developed the capacity to aerobically metabolize citrate.
As this change significantly expanded the food supply available to the \textit{E. coli} population, the change manifested as significantly cloudier cultures.
Experiments plating these cells on media with citrate as the exclusive food source confirmed that, indeed, a Cit$^+$ phenotype had emerged.
Further studies confirmed that cells exhibiting the Cit$^+$ phenotype were descended from the Ara-3 population and were not an artifact of contamination \cite{Blount2008HistoricalColi.}.
The Cit$^+$ phenotype was later revealed to stem from duplication of the gene encoding the CItT transporter under a new promoter, enabling its expression, and therefore the import of citrate, under aerobic conditions \cite{Blount2012GenomicPopulation}.
This late emergence of the Cit$^+$ phenotype in only one replicate lineage, despite the availability of citrate to all twelve LTEE lineages throughout the entire experiment, left researchers curious about the evolutionary path the Ara-3 lineage had traced.

It was hypothesized that so-called potentiating mutations, genetic changes with neutral or unrelated phenotypic consequences that are nonetheless necessary to observe mutations that enable the Cit$^+$ phenotype, had taken place in the Ara-3 lineage
\cite{Blount2016AContingency}.
Using samples suspended at various points in the history of the Ara-3 lineage, experiments plating out trillions of cells to screen for spontaneous mutations enabling Cit$^+$ as well as more extended experimental replays of evolution both support this hypothesis \cite{Blount2008HistoricalColi.}.
The Cit$^+$ phenotype was never observed among offspring from baseline members of the Ara-3 lineage. 
Among  members  of the Ara-3 population several generations before Cit$^+$ was observed to emerge, the Cit$^+$ phenotype was observed to occur  more frequently from members of the specific sublineage from which Cit$^+$ originally emerged than from members of other  contemporary clades.
This led researchers to suspect that at least two potentiating mutations had taken place \cite{Blount2012GenomicPopulation}.
A mutation to the \textit{gltA} gene, which is known to encode citrate synthase, is thought to have served as a potentiating mutation.
This muation improved metabolic function in the presence of acetate (which was generated by cells from a different sublineage).
Coincidentally, it also enabled metabolic benefit to be derived from promoter reassignment of the gene encoding CItT \cite{Blount2016AContingency}.
The identification of other potentiating mutations remains an open question.

\subsection{Evolvability} \label{sec:evolvability}
A significant increase in the rate of mutation occurrence --- a full order of magnitude greater than the baseline REL606 and REL607 populations --- was observed among three LTEE lineages at generation 10,000, Ara$^-2$, Ara$^-4$, and Ara$^+3$  \cite{Sniegowski1997EvolutionColi}.
Injection of plasmids containing wild type copies of seven known mutator locis into cells from these strains were able to rescue mutation rate to its previous lower state.
In such a manner, the high rate of mutation occurrence was ascertained to stem from defects in the methyl-directed mismatch repair pathway in these lines \cite{Sniegowski1997EvolutionColi}.
It was hypothesized that mutator alleles had ``hitch-hiked'' to ubiquity due to their association with unrelated mutations that conferred a significant fitness benefit.

Subsequently, a hypermutator phenotype stemming from a frame-shift mutation to the gene \textit{mutT} was observed to emerge in the Ara–1 population of the LTEE around generation 20,000.
Over the course of the next several thousand generations, alterations to the gene \textit{mutY}, which somewhat reduced the high mutation rate introduced by mutation to \textit{mutT} were observed in two sub-lineages of the Ara-1 population.
Researchers posit that this case might follow a pattern where higher mutation rates are favored by natural selection for a time but are later replaced by lower mutation rates as populations become well adapted to their environments \cite{Wielgoss2013MutationLoad.}.

\subsection{Niche Specialization} \label{sec:niche}
Around generation 6,000, a distinct pair of phenotypes began to be observed among colonies plated up from the LTEE population Ara-2. 
Some cells quickly formed large colonies while others formed only small colonies, even after the passage of an extended period of time.
The former were deemed L cells and the latter, S cells.
Phylogenetic analysis revealed that these two groups were monophyletic.
That is, a divergent pair of phenotypically differentiated but coexisting lineages were observed.
These lineages persisted together for 12,000 generations \cite{Rozen2005Long-termPolymorphism}.

The long duration in which these alternate phenotypes persisted led researchers to hypothesize that frequency-dependent selection was at play.
Indeed, it was shown that although L grows faster, S persists better after exhaustion of the daily glucose supply in the media.
Additionally, it was discovered that the growth of S is promoted by a metabolite excreted by L \cite{Rozen2005Long-termPolymorphism}.
Given the extreme simplicity of the experimental environment that the LTEE populations inhabit, the long-term coexistence of alternate survival strategies --- an apparent emergence of niches --- is a surprising result.